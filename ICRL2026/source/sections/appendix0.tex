\section{Data Preprocessing}
\label{sec:Preprocessing}

In stage {1. Data Preprocessing} (\autoref{fig:CommonStage}), gesture, speech, and text data are read and processed to be represented as vectors or matrices containing information derived from raw data.

For \textbf{text data}: we use the $\texttt{nltk}$ library for tokenization and $\texttt{contractions}$ to normalize contracted words.

One of our contributions is converting the speech data available in ZeroEGGS using Adobe Speech To Text, then align the phonetic timestamps using the Montreal Forced Aligner \citep{saxon2020robust} with an English phoneme dictionary to match the gesture frame rate, generating TextGrid files. From these TextGrids containing word-level timing information, we use $\texttt{gensim}$ to generate word2vec embeddings.

\textbf{Gesture data} consists of BVH (BioVision Motion Capture) files captured via motion capture systems. BVH files include two components: Hierarchy and Motion. Specifically:

\begin{itemize}
	\item \texttt{HIERARCHY}: defines a skeletal tree containing 75 bones $\{ \mathbf{b}_1, \mathbf{b}_2 \cdots \mathbf{b}_{75} \}$, each with an initial position (\texttt{OFFSET}) and \texttt{CHANNELS} parameters specifying the type and order of rotation angles (\texttt{Zrotation}, \texttt{Yrotation}, \texttt{Xrotation}) and position (\texttt{Xposition}, \texttt{Yposition}, \texttt{Zposition}), which are defined in the \texttt{MOTION} section. The first bone (usually \texttt{Hips}) is the root bone $\mathbf{b}_{\text{root}}$, used to define the T-pose via forward kinematics as the initial pose of the skeleton before applying motion.
	
	\item \texttt{MOTION}: a sequence of frames, each containing motion data representing changes of all $75$ bones as defined by the \texttt{CHANNELS} in the \texttt{HIERARCHY}.
\end{itemize}

Our model converts Euler rotation angles to quaternion rotation angles, where a quaternion is a 4-dimensional vector.

\begin{equation} \label{eq:gesturevector}
	\mathbf{g} = \Big[ \mathbf{p}_{\text{root}},  \mathbf{r}_{\text{root}},
	\mathbf{ p }'_{\text{root}},  \mathbf{r}'_{\text{root}},
	\mathbf{p}_{\text{joins}},  \mathbf{r}_{\text{joins}},
	\mathbf{p}'_{\text{joins}},  \mathbf{r}'_{\text{joins}},
	\mathbf{d}_{\text{gaze}}
	\Big]
\end{equation}

Here, each $\mathbf{g} \in \mathbb{R}^{1141}$ includes:
{
	\begin{itemize}
		\item $\mathbf{p}_{\text{root}} \in \mathbb{R}^3$: coordinates of the root joint
		\item $\mathbf{r}_{\text{root}} \in \mathbb{R}^4$: rotation (quaternion) of the root joint
		\item $\mathbf{p}'_{\text{root}} \in \mathbb{R}^3$: velocity of the root position
		\item $\mathbf{r}'_{\text{root}} \in \mathbb{R}^3$: angular velocity of the root rotation
		
		\item $\mathbf{p}_{\text{joins}} \in \mathbb{R}^{3 n_{\text{join} }}$: positions of other joints
		\item $\mathbf{r}_{\text{joins}} \in \mathbb{R}^{6 n_{\text{join} }}$: joint rotations on the X and Y planes
		\item $\mathbf{p}'_{\text{joins}} \in \mathbb{R}^{3n_{\text{join} }}$: velocity of joint positions
		\item $\mathbf{r}'_{\text{joins}} \in \mathbb{R}^{3n_{\text{join} }}$: angular velocity of joint rotations
		\item $\mathbf{d}_{\text{gaze}} \in \mathbb{R}^3$: gaze direction
\end{itemize}}

The original gesture sequences in Euler angles are converted to radians, then converted from Euler to Quaternion as detailed in \autoref{appendix:BVHQuaternion}.

\textbf{Speech data}: $\mathbf{a}_{\text{raw}} \in \mathbb{R}^{ \text{length } }$ is raw speech sampled at 16000 Hz, trimmed into segments $\mathbf{a} \in \mathbb{R}^{64000}$ corresponding to 4 seconds. The paper uses \texttt{ffmpeg-normalize} to normalize volume to a level lower than the original.

\textbf{Emotion}: Emotion data is represented using a predefined one-hot encoded vector. During sampling, the filename encodes the target emotion.

All data is stored using the $\texttt{h5}$ format.
