\chapter*{Tóm tắt}
\label{abstract}

Việc xây dựng người kỹ thuật số (digital human) và các nhân vật trợ lý ảo siêu thực có khả năng tương tác giống như con người đã từ lâu trở thành mục tiêu nghiên cứu của các nhà khoa học máy tính. Sự tiến bộ trong đồ họa máy tính để mô phỏng người siêu thực, cùng với sự phát triển mạnh mẽ của phần cứng và thành công của các mô hình ngôn ngữ lớn, đã mở ra nhiều cơ hội mới. Tuy nhiên, thách thức lớn nhất hiện nay là sinh ra các biểu cảm khuôn mặt (facial expressions) và chuyển động của khung xương 3D  trước khi hiển thị trên màn hình cho người dùng. 
Sinh cử chỉ (gesture generation) không chỉ quan trọng trong việc phát triển các trợ lý ảo mà còn có ứng dụng trong game, robot và ngành công nghiệp điện ảnh. Để giải quyết vấn đề đồng bộ giữa âm thanh và cử chỉ, chúng tôi đề xuất phương pháp \textbf{DeepGesture}, dựa trên mô hình tự động gỡ rối với đầu vào là chuỗi cử chỉ khởi tạo và đoạn âm thanh tương ứng.


%Xây dựng người kỹ thuật số (digital human) hay các nhân vật trợ lý ảo siêu thật có thể tương tác như một con người là mục tiêu nghiên cứu từ lâu của các nhà khoa học máy tính. Với sự phát triển của đồ họa máy tính trong việc mô phỏng người siêu thật cũng như sự phát triển mạnh mẽ của phần cứng máy tính và thành công của các mô hình ngôn ngữ lớn gần đây.
%Thì nút thắt cổ chai duy nhất hiện nay là việc sinh ra các biểu cảm khuôn mặt (facial expression), cũng như các khung xương (joins) 3D trước khi kết xuất (render) để hiển thị lên màn hình cho người dùng.
%Sinh cử chỉ (gesture generation) không chỉ được sử dụng để xây dựng người trợ lý ảo mà còn được sử dụng trong game, robot cũng như trong công nghiệp điện ảnh.
%Để giải quyết vấn đề đồng bộ giữa âm thanh và cử chỉ, chúng tôi đề xuất phương pháp \textbf{OHGesture} dựa trên mô hình diffusion với đầu vào là chuỗi cử chỉ khởi tạo, và đoạn âm thanh tương ứng.	
%Đóng góp của tôi là sử dụng văn bản và cảm xúc, sử dụng cơ chế cross-attention và self-attention trong quá trình diffuse để sinh cử mượt mà và chân thực.
%
%Thực nghiệm chứng minh phương pháp của chúng tôi có thể giúp chúng tôi tạo ra cử chỉ theo nhiều cảm xúc khác nhau và đạt kết quả tốt hơn khi áp dụng cross-attention.

% Ngoài ra mô hình OHGesture cũng sinh được các cử chỉ với đa dạng cảm xúc khác nhau như vui vẻ, buồn bã, người già.

% OHGesture sẽ biểu diễn các dự liệu đầu vào thành các vùng trong không gian với mỗi vùng một đại diện tương ứng. Sau đó sắp xếp lại vị trí đại diện của mỗi vùng dựa trên khoảng cách với cử chỉ khởi tạo để tạo ra các chuỗi cử chỉ ứng viên.

% Cử chỉ sinh ra do người nói có tính tuần hoàn, dựa vào phương pháp mã hóa tuần hoàn (Periodic Autoencoders \cite{starke2022deepphase} ), mô hình sẽ trích xuất được các pha (phrase) của cử chỉ, từ đó giúp mô hình chọn được cử chỉ có pha phù hợp với ngữ nghĩa hoặc nhịp điệu của lời nói.
% Đóng góp của chính của chúng tôi là 

Mã nguồn của chúng tôi được công khai ở

\href{https://github.com/DeepGesture/DeepGesture}{DeepGesture/DeepGesture}